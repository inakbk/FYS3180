\documentclass[11pt,a4wide]{article}
\usepackage{verbatim}
\usepackage{listings}
\usepackage{graphicx}
\usepackage{a4wide}
\usepackage{color}
\usepackage{amsmath}
\usepackage{amssymb}
\usepackage[dvips]{epsfig}
\usepackage[T1]{fontenc}
\usepackage{cite} % [2,3,4] --> [2--4]
\usepackage{shadow}
\usepackage{hyperref}

\setcounter{tocdepth}{2}

\usepackage[labelfont=bf]{caption} %for bold captions (figure, table...)

\lstset{language=c++}
\lstset{alsolanguage=[90]Fortran}
\lstset{basicstyle=\small}
\lstset{backgroundcolor=\color{white}}
\lstset{frame=single}
\lstset{stringstyle=\ttfamily}
\lstset{keywordstyle=\color{red}\bfseries}
\lstset{commentstyle=\itshape\color{blue}}
\lstset{showspaces=false}
\lstset{showstringspaces=false}
\lstset{showtabs=false}
\lstset{breaklines}

\title{{\Huge {\bf Cool Title}}\linebreak \linebreak \small{Project in FYS-3180 at} \linebreak \Large{ Oslo Cyclotron Laboratory}  }
\author{Ina K. B. Kullmann}
\date{\today}

\begin{document}
\maketitle

\begin{abstract}
abstract abstract abstract abstract abstract abstract abstract abstract abstract abstract abstract abstract abstract abstract abstract abstract abstract abstract abstract abstract abstract abstract abstract abstract abstract abstract abstract abstract abstract abstract abstract abstract abstract abstract abstract abstract abstract abstract abstract abstract abstract abstract abstract abstract abstract abstract abstract abstract abstract abstract abstract abstract abstract abstract 

bla bla bla bla bla bla bla bla bla bla bla bla bla bla bla bla bla bla bla bla bla bla bla bla bla bla bla bla bla bla bla bla bla bla bla bla bla bla bla bla bla bla bla 

bla bla bla bla bla bla bla bla bla bla bla bla bla bla bla bla bla bla bla bla bla bla bla bla bla bla blav bla bla bla bla bla bla bla bla bla bla bla bla bla bla bla bla bla bla bla bla bla bla bla bla bla bla bla bla bla bla bla bla bla bla blav bla bla bla bla blav bla bla bla bla bla
\end{abstract}

\begin{figure}[htp]
\centering
\includegraphics[width=0.5\textwidth]{cactus_forside.jpg}
%\caption{The CACTUS detector in the Oslo Cyclotron Laboratory for the study of particle-$\gamma$ coincidences. From \texttt{http://www.mn.uio.no/fysikk/english/research/about/infrastructure/OCL/} }
\label{fig:front_page}
\end{figure}

\newpage

\tableofcontents
\newpage

\section{Motivation and purpose}
%Lucia: say what we have done, i.e, we have analysed data from an experiment with a 16MeV proton beam on a Si target and we want to use the data to get a matrix that shows the gamma rays emitted by 28 Si for a given excitation energy. Briefly say the contents of the report.

The purpose of this article is to give an (brief) introduction to detectors, systems and methods used in experimental nuclear physics. Important because blablabla learning about how to prepare and conduct an experiment and most importantly analyze and interpret the possible error sources.

Will give an short introduction to the OCL
In this project we will focus on the basics of how the cyclotron works and 

We will study the raw data from an previous experiment and analyze it as it was the first time to do so. 

%One of the main projects at OCL is the study of level densities and radiative strength functions. These quantities are important for the understanding of thermodynamic and electromagnetic properties of the atomic nucleus. Also these studies are essential for the understanding of stellar evolution, as well as accelerator-driven transmutation of nuclear waste. (hentet fra nettsiden)

%A major motivation for studying the atomic nucleus is to gain a fundamental understanding of our world, including its origin and future, as well as its current state. Nuclear physics can explain how stars continually work to release virtually all of the useful energy in the world, while at the same time assembling the various elements. Thus today there is a strong collaboration between the fields of nuclear physics and astrophysics.

%There are many potential applications of nuclear physics, e.g. in energy production, medical diagnosis and treatments. There are still several challenges, which make nuclear physics a very interesting and active field of future research. (https://www.mn.uio.no/fysikk/english/research/groups/nuclear/)

We will choose a particular reaction and prepare as for a real experiment by calculating (....energy lost in the ...kin...). We will then use data from an earlier experiment, analyse it and discuss possible error sources(?). Will also vertify/compare data with exsisting databases(?).

We will learn the terms: prompt time, particlebananas, thicknessspectra ++ (?)

Goal: particle-gamma coincidence matrix

%------ noe:
%We know the energy of the beam, the Q value, the final energy of the emitted particles and their angle. We can therefore get the excication energy of the final nucleus

%introduce to a general reaction and Q val so easy to talk about siri/cactus

%then how to /what to measure

\section{Experimental setup and method}
\subsubsection*{The basic consepts of a cyclotron}
A cyclotron is a particle accelerator for charged particles. The particles are accelerated with an external electric field and together with a magnetic field the particles are contained in an orbit inside the cyclotron. In nuclear physics a cyclotron is used to accelerate charged particles so that they leave the cyclotron with the desired energy. The goal is then to study nuclear reactions that occur when the particle beam is directed to a target. Different detectors are used to meashure particles and $\gamma$-rays that are produced in the reaction. 

A simple cyclotron consists of two half -cylinders placed side by side as in figure \ref{fig:cyclotron_draw}.
\begin{figure}[htp]
\centering
\includegraphics[width=0.3\textwidth]{cyclotron_draw.jpg}
\caption{A simple illustration of a cyclotron. The illustration is taken from \texttt{http://www.mn.uio.no/fysikk/english/research/about/infrastructure/OCL/ocl-photos/}}
\label{fig:cyclotron_draw}
\end{figure}
Every time the particles pass between the two cylinders they are accelerated by an oscillating electric field. Therefore the particles increase speed and radius for every half round. Inside the cylinders the electric field is zero, but there is a magnetic field perpendicular to the plane showed in figure \ref{fig:cyclotron_draw} that contain the particles in a circular orbit. When the radius of the particle beam is bigger than the radius of the cylinders the particles leave the cyclotron. 

\subsection{The Oslo Cyclotron laboratory (OCL)}
The Oslo Cyclotron Laboratory (OCL) houses the only accelerator in Norway for ionized atoms in basic research\footnote{http://www.mn.uio.no/fysikk/english/research/about/infrastructure/OCL/index.html}. The accelerator is used in various fields of research for instance nuclear physics and nuclear chemistry. Other applications for the Cyclotrone are the production of isotopes for nuclear medicine. The reasearch in nuclear physics at Oslo Cyclotron Laboratory mainly focus on studying the level densities and radiative strength functions where the overall goal is to better understand the atomic nuclei. 

\begin{figure}[htp]
\centering
\includegraphics[width=0.6\textwidth]{ocl-layout_mini.jpg}
\caption{An overview of the Oslo Cyclotron Laboratory with the experimental hall to the right with the cyclotron at the bottom right. The beam line are indicated with a blue line and the target chamber is at the top left (CACTUS/SiRi). The possible beam types, energy and intensity ranges are indicated in the table to bottom left. }
\label{fig:OLC_exp_hall}
\end{figure}

An overview of the Oslo Cyclotron Laboratory is given in figure \ref{fig:OLC_exp_hall}. The possible beam types, energy and intensity ranges are indicated in the table to bottom left. In figure \ref{fig:OLC_exp_hall} we can see the cyclotron vault to the far right with the cyclotron (MC-35 Scanditronix Cyclotron) at the bottom right. The beam of the accelerated particles travels first from the cyclotron along the beam line through a switching magnet and then to a analyzing magnet. The analyzing magnet directs the beam out of the cyclotron vault and into the experimental hall by turning the beam 90 degrees. Then the beam goes through another swiching magnet before hitting the target chamber (CACTUS/SiRi) to the far left in figure \ref{fig:OLC_exp_hall}. Around the target chamber there are two detectors, CACTUS and SiRi. The swiching magnets can also direct the beam to different target stations, but we will only have a closer look at the target chamber associated to the CACTUS and SiRi arrays. 

%deflectors ? slits ?

\subsubsection{The CACTUS and SiRi detectors}
The CACTUS/SiRi detector can be used to study particle-gamma coincidences. In figure \ref{fig: cactus_siri} we see an illustration of a particle from the beam hitting a target nucleus. After the reaction a gamma-ray and a particle is emitted in addition to the resulting nucleus beeing changed. We see that the gamma is measured by the CACTUS detector and the emitted particle by the SiRi detector. The figure indicates that the angle between the incident trajectory and the trajectory of the emmitted particle is given as $\theta$.
\begin{figure}[htp]
\centering
\includegraphics[width=0.5\textwidth]{cactus_siri.png}
\caption{A incident particle hitting a target nucleus. The resulting emmited $gamma$-ray is detected by the CACTUS detectors and the emmitted particle is detected by the SiRi detector. The angle between the incident trajectory and the trajectory of the emmitted particle is given as $\theta$. The two parts of the SiRi detector, 'dE' and 'E' is indicated in the figure.}
\label{fig: cactus_siri}
\end{figure}

When looking at the front picture, figure \ref{fig:front_page}, it is not hard to imagine where the CACTUS detector have gotten its name from. The CACTUS detector meashures the energies of the $\gamma$-rays and counts the number of $\gamma$-rays. The detector consists of 28 NaI scintillation detectors spherically distributed around the target chamber, pointing out like a Cactus. Each of the NaI scintillation detectors  meashure the energy of the $\gamma$-radiation by using the excitation effect of the incident radiation on a scintillator material (NaI). When the scintillator is excited by radiation it produces a signal that is then converted into an electrical signal that the electronics of the detector process\footnote{https://en.wikipedia.org/wiki/Scintillation\_counter}. 

The SiRi-array measures the energy of the resulting emitted particle and consists of 8 Silicon detectors on a ring. Each detector is divided into 8 strips which also makes it possible to also measure the angle of the particle. The Si detectors uses the properties of a semiconductor, doped Silicon, to meashure the path and energy of the charged particles by detecting the small ionization currents that occur when the charged particles move through the material\footnote{https://en.wikipedia.org/wiki/Semiconductor\_detector}. In figure \ref{fig: siri} we see the Silicon Ring (SiRi) to the left and a illustration of one of the detectors on the right with the induvidual strips marked.
\begin{figure}[htp]
\centering
\includegraphics[width=0.8\textwidth]{siri.png}
\caption{The SiRi detector used to measure the energy of a particle from a particle-gamma coincidence. Left: A picture of Tthe Silicon Ring (SiRi). Right: A drawing of one of the 8 detectors on ring with the induvidual strips marked.}
\label{fig: siri}
\end{figure}

The SiRi detector stops the emmited particle, so it looses all its energy as it moves trough the material. The detector is divided into two parts, one called 'dE' and the other simply 'E'. The first part 'dE' is 130 micrometers thick and this is where the particle looses some of its energy. In the other part 'E' the particle looses the remaining energy and stops. In addition, an Aluminium foil of 2.8mg/cm${}^2$ thickness is placed before the dE detector. The 'dE' and 'E' positions are indicated in figure \ref{fig: cactus_siri}.

%Explain which sort of experiments we do in the lab: how do we measure emitted charged particles and gammas in coincidence, using SiRi and CACTUS. Explain that we can reconstruct the excitation energy levels of a nucleus from the measured charged particles and that we use this to build a coincidence matrix (related to probability of nuclear decay from excitation energy Ex with gammma energy Egamma). 

%Briefly describe what SiRi and CACTUS are, and how we can use SIRI to plot the particle bananas and distinguish between the different kinds of particles. You might want to say what a scintillator is, and briefly define how the Si detector works.

\subsection{Choice of reaction}% and preparations before the experiment}
In this project we have chosen the reaction ${}^{28}\mathrm{Si(p,p')}^{28}\mathrm{Si}$ drawn in figure \ref{fig: reaction}. The incident proton will have an energy of 16MeV and the target of ${}^28$Si will have a thickness of 4mg/cm${}^2$. 

\begin{figure}[htp]
\centering
\includegraphics[width=0.4\textwidth]{reaction.png}
\caption{An illustration of the chosen ${}^{28}\mathrm{Si(p,p')}^{28}\mathrm{Si}$ reaction.}
\label{fig: reaction}
\end{figure}

%Before conducting the experiment we will study how the proton beam loses energy as it goes through the target and the other materials, until it reaches the E detector. To do this we will use the \texttt{kin} software. (include kin?)

%Explain that we have codes  (KIN) to simulate how the charged particles lose energy as they travel through the target, dtectors, etc. Write down the example done in class and the results. ​

%How much energy does it lose in each step? The target is 4mg/cm2 thickness, and the dE detector is 130 micrometers thick. In addition, an Al foil of 2.8mg/cm2 thickness is placed before the dE detector. The protons are scattered 48 degrees with respect to the beam direction.

\section{Data analysis of experimental data}
The data stored from the experiment are the 'dE' and 'E' signals of the charged particles measured by SiRi and the energy of the $\gamma$-rays in coincidence with the charged particles, measured by the CACTUS detector. The data collected with the CACTUS and SiRi detectors are stored in event files; large files where each measured parameter from each event is written down. There are millions of event files that have to be analyzed to be able to extract information from the experiment. Luckily the OLC labratory have written a sorting code which does the sorting of the event files. This sorting code needs to be run before the data can be plotted. %We will also have to correct the data for different effects and calibrate (?)

noenoenoe ?

\subsection{Particle calibration and bananas}
%explain what it means and show the effect in the bananas.
First we have to calibrate the particle detectors, or the SiRi-array. This is done by plotting 'dE' versus 'E', obtaining curves commonly known as 'bananas'. In figure \ref{fig: de_e} the uncalibrated and calibrated bananas are shown. 

The bananas are characteristic for each type of ejected particle. We can use this to...??????

\begin{figure}[htp]
\centering
\includegraphics[width=0.4\textwidth]{m_e_de_uncalibrated.pdf}
\includegraphics[width=0.4\textwidth]{m_e_de_calibrated.pdf} %noe feil her!?!
\caption{The 'particle bananas', or 'dE' versus 'E' plotted. Left: uncalibrated. Right: calibrated.}
\label{fig: de_e}
\end{figure}

\subsection{Selecting a reaction}
When the particle calibration is done we can select a reaction, gate on the banana corresponding to the emitted protons. We wil use this to get information about ${}^28$Si. 

In figure \ref{fig: banana_gate} we see the gate chosen on one banana

\begin{figure}[htp]
\centering
\includegraphics[width=0.4\textwidth]{m_e_de_thick.pdf}
\caption{The gating on one particle banana.}
\label{fig: banana_gate}
\end{figure}

%explain how we gated on the banana corresponding to the emitted protons to get info about 28Si. As a result, we obtain the excitation energies of the nucleus. Show the plot and compared with the data from the nndc website.

%Set a gate on the ejected protons. Get the particle spectra.
%2.5 Set a gate on an excited state. How can this help to the gamma calibration?
%2.6. Learn how to export matrices into mama format and look at them in mama. Which excitation energy levels you can see?
%2.7 Compare your results with the Table of Isotopes ( or with the data from the chart of nuclides in http://www.nndc.bnl.gov/chart/ )

??????

\subsection{$\gamma$-calibration}

%is this one actually for next subsection?
\begin{figure}[htp]
\centering
\includegraphics[width=0.4\textwidth]{m_e_de_uncalibrated.pdf}
\includegraphics[width=0.4\textwidth]{m_nai_e_t_corrected_fit.pdf}
\caption{The, Left: uncalibrated. Right: calibrated.}
\label{fig: time_corr}
\end{figure}


\subsection{Treatment of the time signals}
%- Explain that we first aligned them and show the plot of the m_nai signals aligned. Explain that we have leading edge discriminators, and therefore we need to correct from "walk". Show the signals corrected and explain how you did this.
%- Selecting coincidences: show the plot and explain what is the prompt peak, what is the time resolution, and how we can calculate the time between beam pulses from the plot. Explain why we need to gate on the prompt peak.

\subsection{ The coincidence matrix}
%Show the coincidence matrix. Mention that there is background, how it can be measure and that it was subtracted. Show the diagonal Eg=Ex in the matrix. Why do we see some counts outside? Explain the peaks we see and comp.re them to the results from the nndc website or the TOI.

\section{The Oslo method}
noe noe tar lang tid

\subsection{Unfolding of the coincidence matrix}
%Explain the interactions of gamma with matter, and what does unfolding mean.
%Show the unfolded matrix.

\subsection{Multiplicity}


not gotten eny results really, need more analyzing to get any
\section{Discussion and Experiences}


\noindent\rule{\textwidth}{1pt}






\end{document}





