\documentclass[11pt,a4wide]{article}
\usepackage{verbatim}
\usepackage{listings}
\usepackage{graphicx}
\usepackage{a4wide}
\usepackage{color}
\usepackage{amsmath}
\usepackage{amssymb}
\usepackage[dvips]{epsfig}
\usepackage[T1]{fontenc}
\usepackage{cite} % [2,3,4] --> [2--4]
\usepackage{shadow}
\usepackage{hyperref}

\setcounter{tocdepth}{2}

\lstset{language=c++}
\lstset{alsolanguage=[90]Fortran}
\lstset{basicstyle=\small}
\lstset{backgroundcolor=\color{white}}
\lstset{frame=single}
\lstset{stringstyle=\ttfamily}
\lstset{keywordstyle=\color{red}\bfseries}
\lstset{commentstyle=\itshape\color{blue}}
\lstset{showspaces=false}
\lstset{showstringspaces=false}
\lstset{showtabs=false}
\lstset{breaklines}

\title{{\Huge {\bf Cool Title}}\linebreak \small{Project 2, FYS-3150}}
\author{Ina K. B. Kullmann}
\date{\today}

\begin{document}
\maketitle

{\scriptsize \noindent All source codes can be found at: \texttt{https://github.com/inakbk/Project\_1.git} in the folders \texttt{exercise\_b} and \texttt{exercise\_d}. }

\tableofcontents
\newpage

\begin{abstract}
bla bla bla bla bla bla bla bla bla bla bla bla bla bla bla bla bla bla bla bla bla bla bla bla bla bla bla bla bla bla bla bla bla bla bla bla bla bla bla bla bla bla bla bla bla bla bla bla bla bla bla bla bla bla bla bla bla bla bla bla bla bla bla bla bla bla bla bla bla blav bla bla bla bla bla bla bla bla bla bla bla bla bla bla bla bla bla bla bla bla bla bla bla bla bla bla bla bla bla bla bla bla bla bla blav bla bla bla bla blav bla bla bla bla bla

bla bla bla bla bla bla bla bla bla bla bla bla bla bla bla bla bla bla bla bla bla bla bla bla bla bla bla bla bla bla bla bla bla bla bla bla bla bla bla bla bla bla bla bla bla bla bla bla bla bla bla bla bla bla bla bla bla bla bla bla bla bla bla bla bla bla bla bla bla blav bla bla bla bla bla bla bla bla bla bla bla bla bla bla bla bla bla bla bla bla bla bla bla bla bla bla bla bla bla bla bla bla bla bla blav bla bla bla bla blav bla bla bla bla bla
\end{abstract}


\section{Motivation and purpose}
The aim of this project is to solve Schr\"odinger's equation for two electrons in a three-dimensional harmonic oscillator well with and without a repulsive Coulomb interaction.  Your task is to solve this equation by reformulating it in a discretized form as an eigenvalue equation to be solved with Jacobi's method. -->code which implements Jacobi's method.

Electrons confined in small areas in semiconductors, so-called quantum dots, form a hot research area in modern solid-state physics, with applications spanning from such diverse fields as quantum nano-medicine to the contruction of quantum gates. You can read about quantum dots at \url{http://en.wikipedia.org/wiki/Quantum_dot}, which also contains links to several scientific articles. A recent article of interest is the review by Semonin {\em et al} in Materials Today, volume 15, page 508 (2012).

In this article we will let two electrons move in a three-dimensional harmonic oscillator potential that repel each other via the Coulomb interaction.  Throughout this paper we will assume spherically symmetry and let the angular momentum be $l=0$. Before solving the problem for two electrons we will look at one electron in a harmonic oscillator potential. 

To solve the Schr\"odinger's equation for one electron(and two??) we will have to transformed the equations into a matrix eigenvalue problem. When we have rewritten the problem we will use Jacobi's rotation algorithm to find the solutions. % (see Lecture notes chapter 7)   .

Together with linear equations and least squares, the third major problem in matrix com- putations deals with the algebraic eigenvalue problem. Here we limit our attention to the symmetric case. We focus in particular similarity transformations, --> Jacobi \footnote{sitert fra lectures2015 p. 225}


\section{Solving eigenvalueproblems using similarity transformations}



\noindent\rule{\textwidth}{1pt}






\end{document}





