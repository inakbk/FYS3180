\documentclass[11pt,a4wide]{article}
\usepackage{verbatim}
\usepackage{listings}
\usepackage{graphicx}
\usepackage{a4wide}
\usepackage{color}
\usepackage{amsmath}
\usepackage{amssymb}
\usepackage[dvips]{epsfig}
\usepackage[T1]{fontenc}
\usepackage{cite} % [2,3,4] --> [2--4]
\usepackage{shadow}
\usepackage{hyperref}

\setcounter{tocdepth}{2}

\usepackage[labelfont=bf]{caption} %for bold captions (figure, table...)

\lstset{language=c++}
\lstset{alsolanguage=[90]Fortran}
\lstset{basicstyle=\small}
\lstset{backgroundcolor=\color{white}}
\lstset{frame=single}
\lstset{stringstyle=\ttfamily}
\lstset{keywordstyle=\color{red}\bfseries}
\lstset{commentstyle=\itshape\color{blue}}
\lstset{showspaces=false}
\lstset{showstringspaces=false}
\lstset{showtabs=false}
\lstset{breaklines}

\title{{\Huge {\bf Cool Title}}\linebreak \linebreak \small{Project in FYS-3180 at} \linebreak \Large{ Oslo Cyclotron Laboratory}  }
\author{Ina K. B. Kullmann}
\date{\today}

\begin{document}
\maketitle

\begin{abstract}
abstract abstract abstract abstract abstract abstract abstract abstract abstract abstract abstract abstract abstract abstract abstract abstract abstract abstract abstract abstract abstract abstract abstract abstract abstract abstract abstract abstract abstract abstract abstract abstract abstract abstract abstract abstract abstract abstract abstract abstract abstract abstract abstract abstract abstract abstract abstract abstract abstract abstract abstract abstract abstract abstract 

bla bla bla bla bla bla bla bla bla bla bla bla bla bla bla bla bla bla bla bla bla bla bla bla bla bla bla bla bla bla bla bla bla bla bla bla bla bla bla bla bla bla bla 

bla bla bla bla bla bla bla bla bla bla bla bla bla bla bla bla bla bla bla bla bla bla bla bla bla bla blav bla bla bla bla bla bla bla bla bla bla bla bla bla bla bla bla bla bla bla bla bla bla bla bla bla bla bla bla bla bla bla bla bla bla blav bla bla bla bla blav bla bla bla bla bla
\end{abstract}

\begin{figure}[htp]
\centering
\includegraphics[width=0.5\textwidth]{cactus_forside.jpg}
\caption{The CACTUS detector in the Oslo Cyclotron Laboratory for the study of particle-$\gamma$ coincidences. (\texttt{//www.mn.uio.no/fysikk/english/research/about/infrastructure/OCL/}) }
\label{fig:front_page}
\end{figure}

\newpage

\tableofcontents
\newpage

\section{Motivation and purpose}
%Lucia: say what we have done, i.e, we have analysed data from an experiment with a 16MeV proton beam on a Si target and we want to use the data to get a matrix that shows the gamma rays emitted by 28 Si for a given excitation energy. Briefly say the contents of the report.

The purpose of this article is to give an (brief) introduction to detectors, systems and methods used in experimental nuclear physics. Important because blablabla learning about how to prepare and conduct an experiment and most importantly analyze and interpret the possible error sources.

Will give an short introduction to the OCL
In this project we will focus on the basics of how the cyclotron works and 

We will study the raw data from an previous experiment and analyze it as it was the first time to do so. 

%One of the main projects at OCL is the study of level densities and radiative strength functions. These quantities are important for the understanding of thermodynamic and electromagnetic properties of the atomic nucleus. Also these studies are essential for the understanding of stellar evolution, as well as accelerator-driven transmutation of nuclear waste. (hentet fra nettsiden)

%A major motivation for studying the atomic nucleus is to gain a fundamental understanding of our world, including its origin and future, as well as its current state. Nuclear physics can explain how stars continually work to release virtually all of the useful energy in the world, while at the same time assembling the various elements. Thus today there is a strong collaboration between the fields of nuclear physics and astrophysics.

%There are many potential applications of nuclear physics, e.g. in energy production, medical diagnosis and treatments. There are still several challenges, which make nuclear physics a very interesting and active field of future research. (https://www.mn.uio.no/fysikk/english/research/groups/nuclear/)

We will choose a particular reaction and prepare as for a real experiment by calculating (....energy lost in the ...kin...). We will then use data from an earlier experiment, analyse it and discuss possible error sources(?). Will also vertify/compare data with exsisting databases(?).

We will learn the terms: prompt time, particlebananas, thicknessspectra ++ (?)

Goal: particle-gamma coincidence matrix

%------ noe:
%We know the energy of the beam, the Q value, the final energy of the emitted particles and their angle. We can therefore get the excication energy of the final nucleus

%introduce to a general reaction and Q val so easy to talk about siri/cactus

%then how to /what to measure

\section{Experimental setup and method}
\subsubsection*{The basic consepts of a cyclotron}
A cyclotron is a particle accelerator for charged particles. The particles are accelerated with an external electric field and together with a magnetic field the particles are contained in an orbit inside the cyclotron. In nuclear physics a cyclotron is used to accelerate charged particles so that they leave the cyclotron with the desired energy. The goal is then to study nuclear reactions that occur when the particle beam is directed to a target. Different detectors are used to meashure particles and $\gamma$-rays that are produced in the reaction. 

A simple cyclotron consists of two half -cylinders placed side by side as in figure \ref{fig:cyclotron_draw}.
\begin{figure}[htp]
\centering
\includegraphics[width=0.3\textwidth]{cyclotron_draw.jpg}
\caption{A simple illustration of a cyclotron. The illustration is taken from \texttt{http://www.mn.uio.no/fysikk/english/research/about/infrastructure/OCL/ocl-photos/}}
\label{fig:cyclotron_draw}
\end{figure}
Every time the particles pass between the two cylinders they are accelerated by an oscillating electric field. Therefore the particles increase speed and radius for every half round. Inside the cylinders the electric field is zero, but there is a magnetic field perpendicular to the plane showed in figure \ref{fig:cyclotron_draw} that contain the particles in a circular orbit. When the radius of the particle beam is bigger than the radius of the cylinders the particles leave the cyclotron. 

\subsection{The Oslo Cyclotron laboratory (OCL)}
The Oslo Cyclotron Laboratory (OCL) houses the only accelerator in Norway for ionized atoms in basic research\footnote{http://www.mn.uio.no/fysikk/english/research/about/infrastructure/OCL/index.html}. The accelerator is used in various fields of research for instance nuclear physics and nuclear chemistry. Other applications for the Cyclotrone are the production of isotopes for nuclear medicine. The reasearch in nuclear physics at Oslo Cyclotron Laboratory mainly focus on studying the level densities and radiative strength functions where the overall goal is to better understand the atomic nuclei. 

\begin{figure}[htp]
\centering
\includegraphics[width=0.6\textwidth]{ocl-layout_mini.jpg}
\caption{An overview of the Oslo Cyclotron Laboratory with the experimental hall to the right with the cyclotron at the bottom right. The beam line are indicated with a blue line and the target chamber is at the top left (CACTUS/SiRi). The possible beam types, energy and intensity ranges are indicated in the table to bottom left. }
\label{fig:OLC_exp_hall}
\end{figure}

An overview of the Oslo Cyclotron Laboratory is given in figure \ref{fig:OLC_exp_hall}. The possible beam types, energy and intensity ranges are indicated in the table to bottom left. In figure \ref{fig:OLC_exp_hall} we can see the cyclotron vault to the far right with the cyclotron (MC-35 Scanditronix Cyclotron) at the bottom right. The beam of the accelerated particles travels first from the cyclotron along the beam line through a switching magnet and then to a analyzing magnet. The analyzing magnet directs the beam out of the cyclotron vault and into the experimental hall by turning the beam 90 degrees. Then the beam goes through another swiching magnet before hitting the target chamber (CACTUS/SiRi) to the far left in figure \ref{fig:OLC_exp_hall}. Around the target chamber there are two detectors, CACTUS and SiRi. The swiching magnets can also direct the beam to different target stations, but we will only have a closer look at the target chamber associated to the CACTUS and SiRi arrays. 

%deflectors ? slits ?

\subsubsection{The CACTUS and SiRi detectors}
The CACTUS/SiRi detector can be used to study particle-gamma coincidences. In figure \ref{fig: cactus_siri} we see an illustration of a particle from the beam hitting a target nucleus. After the reaction a gamma-ray and a particle is emitted in addition to the resulting nucleus beeing changed. We see that the gamma is measured by the CACTUS detector and the emitted particle by the SiRi detector. The figure indicates that the angle between the incident trajectory and the trajectory of the emmitted particle is given as $\theta$.
\begin{figure}[htp]
\centering
\includegraphics[width=0.5\textwidth]{cactus_siri.png}
\caption{A incident particle hitting a target nucleus. The resulting emmited $gamma$-ray is detected by the CACTUS detectors and the emmitted particle is detected by the SiRi detector. The angle between the incident trajectory and the trajectory of the emmitted particle is given as $\theta$. The two parts of the SiRi detector, 'dE' and 'E' is indicated in the figure.}
\label{fig: cactus_siri}
\end{figure}

When looking at the front picture, figure \ref{fig:front_page}, it is not hard to imagine where the CACTUS detector have gotten its name from. The CACTUS detector meashures the energies of the $\gamma$-rays and counts the number of $\gamma$-rays. The detector consists of 28 NaI scintillation detectors spherically distributed around the target chamber, pointing out like a Cactus. Each of the NaI scintillation detectors  meashure the energy of the $\gamma$-radiation by using the excitation effect of the incident radiation on a scintillator material (NaI). When the scintillator is excited by radiation it produces a signal that is then converted into an electrical signal that the electronics of the detector process\footnote{https://en.wikipedia.org/wiki/Scintillation\_counter}. 

The SiRi-array measures the energy of the resulting emitted particle and consists of 8 Silicon detectors on a ring. Each detector is divided into 8 strips which also makes it possible to also measure the angle of the particle. The Si detectors uses the properties of a semiconductor, doped Silicon, to meashure the path and energy of the charged particles by detecting the small ionization currents that occur when the charged particles move through the material\footnote{https://en.wikipedia.org/wiki/Semiconductor\_detector}. In figure \ref{fig: siri} we see the Silicon Ring (SiRi) to the left and a illustration of one of the detectors on the right with the induvidual strips marked.
\begin{figure}[htp]
\centering
\includegraphics[width=0.8\textwidth]{siri.png}
\caption{The SiRi detector used to measure the energy of a particle from a particle-gamma coincidence. \textbf{Left:} A picture of Tthe Silicon Ring (SiRi). \textbf{Right:} A drawing of one of the 8 detectors on ring with the induvidual strips marked.}
\label{fig: siri}
\end{figure}

The SiRi detector stops the emmited particle, so it looses all its energy as it moves trough the material. The detector is divided into two parts, one called 'dE' and the other simply 'E'. The first part 'dE' is 130 micrometers thick and this is where the particle looses some of its energy. In the other part 'E' the particle looses the remaining energy and stops. In addition, an Aluminium foil of 2.8mg/cm${}^2$ thickness is placed before the dE detector. The 'dE' and 'E' positions are indicated in figure \ref{fig: cactus_siri}.

%Explain which sort of experiments we do in the lab: how do we measure emitted charged particles and gammas in coincidence, using SiRi and CACTUS. Explain that we can reconstruct the excitation energy levels of a nucleus from the measured charged particles and that we use this to build a coincidence matrix (related to probability of nuclear decay from excitation energy Ex with gammma energy Egamma). 

%Briefly describe what SiRi and CACTUS are, and how we can use SIRI to plot the particle bananas and distinguish between the different kinds of particles. You might want to say what a scintillator is, and briefly define how the Si detector works.

\subsection{Choice of reaction}% and preparations before the experiment}
In this project we have chosen the reaction ${}^{28}\mathrm{Si(p,p')}^{28}\mathrm{Si}$ drawn in figure \ref{fig: reaction}. The incident proton will have an energy of 16MeV and the target of ${}^28$Si will have a thickness of 4mg/cm${}^2$. 

\begin{figure}[htp]
\centering
\includegraphics[width=0.4\textwidth]{reaction.png}
\caption{An illustration of the chosen ${}^{28}\mathrm{Si(p,p')}^{28}\mathrm{Si}$ reaction.}
\label{fig: reaction}
\end{figure}

%Before conducting the experiment we will study how the proton beam loses energy as it goes through the target and the other materials, until it reaches the E detector. To do this we will use the \texttt{kin} software. (include kin?)

%Explain that we have codes  (KIN) to simulate how the charged particles lose energy as they travel through the target, dtectors, etc. Write down the example done in class and the results. ​

%How much energy does it lose in each step? The target is 4mg/cm2 thickness, and the dE detector is 130 micrometers thick. In addition, an Al foil of 2.8mg/cm2 thickness is placed before the dE detector. The protons are scattered 48 degrees with respect to the beam direction.

\section{Data analysis of experimental data}
The data stored from the experiment are the 'dE' and 'E' signals of the charged particles measured by SiRi and the energy of the $\gamma$-rays in coincidence with the charged particles, measured by the CACTUS detector. The data collected with the CACTUS and SiRi detectors are stored in event files; large files with the measured parameter from each event. To extract information from the experiment one have to analyze millions of event files. Luckily the OLC labratory have written a sorting code which does the sorting of the event files. In the process the user can choose to include different features to obtain the final result, a 'clean' coincidence matrix.

In the analysis process the following programs have been used:
\begin{itemize}
\item \texttt{Makefile:} executable file that calls for example \texttt{User\_sort.cpp} and creates an executable called \texttt{sorting}.
\item \texttt{User\_sort.cpp:} the main sorting code in C++. It is possible to modify the code to include time gates, gates on excited nuclear states and so on. It defines what the executable \texttt{sorting} will do when it is run. 
\item \texttt{Sorting:}: executable file created by the \texttt{Makefile}. It uses the batch file when run.
\item \texttt{Name.batch:} holds information on where to find the data files to be sorted. 'Name' is usually the experiment reaction. The file also includes several parameters and calls the following two programmes:
\begin{itemize}
\item \texttt{gainshifts525.dat:} contains information about the calibration of the particle and gamma detectors, together with the time signal calibration.
\item \texttt{zrangep.dat:} the range file for the ejected protons. It has information about how the ejected protons lose energy as they penetrate the E dE Si detectors.
\end{itemize}
\end{itemize}
The sorting codes can first be run without calibration parameters ('plain') and later with calibration when the parameters are found. 

%This sorting code needs to be run before the data can be plotted. %We will also have to correct the data for different effects and calibrate (?)

%noenoenoe ?

\subsection{Particle calibration and bananas} \label{sec: particle_calib}
First we have to calibrate the particle detectors, or the SiRi-array. This is done by plotting 'dE' versus 'E', obtaining curves commonly known as 'bananas'. To the left in figure \ref{fig: de_e} we see the uncalibrated plot of the bananas obtained when the sorting routine is run 'plain'. The bananas are characteristic for each type of ejected particle, there is one banana for each particle. On each banana we can see peaks corresponding to the excited states of the particle. 

\begin{figure}[htp]
\centering
\includegraphics[width=0.4\textwidth]{m_e_de_uncalibrated.pdf}
\includegraphics[width=0.4\textwidth]{m_e_de_calibrated.pdf} %noe feil her!?!
\caption{The 'particle bananas', or 'dE' versus 'E' plotted. \textbf{Left:} uncalibrated. \textbf{Right:} calibrated. Each banana corresponds to one emitted particle. We can also see the peacks corresponding to the excited states of the particle.}
\label{fig: de_e}
\end{figure}

The SiRi-array has 8 Silicon detectors, each devided into 8 strips. Theoretically all the bananas should be the same for a given angle. This is not the case experimentally. We correct this experimental effect by aligning all the detectors, or finding the corresponding gain og shift in reference to one detector. We also choose a reference point, for example the energies of the peaks in the bananas corresponding to the ground and first excited states. This reference point can be calculated with the \texttt{kin} software explained earlier (?). %“OCL SiRi Kinematics Calculator”

The response of the detectors should be approximately linear with energy in both $\Delta$E and E meashurements and can be expressed as:
\begin{align*}
E(x) &= a_E + b_Ex\\
\Delta E(x) &= a_\Delta + b_\Delta x
\end{align*}
where $x$ is the channel number, $a$ the shift in the energies and $b$ the gain in the energies. The coefficients $a$ and $b$ are then calculated in such a way that the value $E(x)$ and $\Delta E(x)$ for the reference peaks matches the values calculated with \texttt{kin}. This calibration is performed for all the 8 strips and the 8 detectors and then included in the \texttt{gain\_shift.dat} file. So when the calibration is done the gain-shift file includes the $a$ and $b$ coefficients for both the $\Delta$E and E for each detector.

To the right in figure \ref{fig: de_e} we see the calibrated bananas obtained when the sorting routine is run with the gain-shift file included. We see that the calibrated plot is much clearer and have sharper excited state peaks than the uncalibrated plot. 

\subsection{Selecting the data for the ${}^{28}$Si(p,p')$^{28}$Si reaction}
When the particle calibration is done we want to only select the data from the ${}^{28}$Si(p,p')$^{28}$Si reaction, we need to gate on the banana corresponding to the emitted protons. This is done by using or 'commenting in' the range file \texttt{zrangep.dat} in the \texttt{.batch}-file. In the left plot in figure \ref{fig: banana_gate} we see the aparent thickness of the $\Delta$E detector. The peak is centered at $\approx 130 \mu$m which is the actual thickness of the $\Delta$E detector, and the width or range is $\approx 20\mu$m. The range file includes the peak and width read from the left plot in figure \ref{fig: banana_gate}. 

When we use the range file in the \texttt{.batch}-file the sorting routine distributes the experimental data around the actual thickness of the $\Delta$E detector. This means that we have gated on the banana that corresponds to the emitted proton or the desired reaction ${}^{28}$Si(p,p')$^{28}$Si. The selected data can therefore now give us information about the reaction we want to study. 

In figure \ref{fig: banana_gate} we see the 'banana' plot obtained when running the sorting routine again with the range file. By comparing with the right plot in figure \ref{fig: de_e} we see that we have selected the banana corresponding to the emitted protons. 

\begin{figure}[htp]
\centering 
\includegraphics[width=0.4\textwidth]{h_thick_fromcalibrated_zoom.pdf}
\includegraphics[width=0.4\textwidth]{m_e_de_thick.pdf}
\caption{Selecting the data for the ${}^{28}$Si(p,p')$^{28}$Si reaction. \textbf{Left:} The calculated thickness of the $\Delta$E detector. The peak is centered at $\approx 130 \mu$m which is the actual thickness of the $\Delta$E detector. Other peaks would have corresponded to other ejected particles. \textbf{Right:} The 'banana' plot after particle calibration and gating on one particle banana.}
\label{fig: banana_gate}
\end{figure}

After including the range file and sorting again we can find the excitation levels (of the final nucleus?) by projecting the coincidence matrix on the y axis and compare this with other experimental data. In table \ref{tab: e_levels} we see the excited states of ${}^{28}$Si collected from the database of NNDC \footnote{//www.nndc.bnl.gov/chart/getdataset.jsp?nucleus=28SI\&unc=nds}. 

\begin{table}
\centering
\caption{The energy levels of ${}^{28}$Si from the Chart of Nuclides found on the NNDC website \texttt{//www.nndc.bnl.gov/chart/getdataset.jsp?nucleus=28SI\&unc=nds}}
\begin{tabular}{|c|}
\hline 
E\_x [keV] \\ 
\hline 
0 \\ 
\hline 
1779.030 11 \\ 
\hline 
4617.86 4 \\ 
\hline 
4979.92 8 \\ 
\hline 
6276.2 7 \\ 
\hline 
6690.74 15 \\ 
\hline 
\end{tabular} 
\label{tab: e_levels}
\end{table}

In figure \ref{fig: proj_y_excitation} we see the coincidence matrix projected on the y axis. Each peak in the plot correspond to an excited state of ${}^{28}$Si. The six first excited stated are marked with a red circle. We see that the values of the excited states marked corresponds well to the values of the energy levels found in table \ref{tab: e_levels}. Peaks that do not correspond to a energy level of ${}^{28}$Si is contamination which will be corrected for at a later stage (?). 

\begin{figure}[htp]
\centering
\includegraphics[width=0.4\textwidth]{h_ex_calibrated-zoom.png}
\caption{The the coincidence matrix projected on the y axis giving number of counts versus energy, $E_x$. Each peak corresponds to a excited state of ${}^{28}$Si. The six first excited stated are marked with a red circle to be compared with table \ref{tab: e_levels}. }
\label{fig: proj_y_excitation}
\end{figure}
%y axis here is number of counts, other axis other plots?

%explain how we gated on the banana corresponding to the emitted protons to get info about 28Si. As a result, we obtain the excitation energies of the nucleus. Show the plot and compared with the data from the nndc website.

\subsection{$\gamma$-calibration and treatment of the time signals}
We will now look at the procedure used to calibrate the NaI detectors and the time signals. The CACTUS $\gamma$ detector consists of 28 NaI detectors, and due to differences in the electronics the time signals are not aligned. To correct for this we will use a similar procedure as in section \ref{sec: particle_calib}. We will choose one of the 28 detectors as a reference and read of the peak of (?). We will then use this peak to shift the count(?) peaks for all the other detectors so that all the peaks are aligned. We will also account for the different responce in the detectors, or the gain as in section \ref{sec: particle_calib}.
%could have included m_nai_e plot to show alignment of signals, but what are the axes?

Finally we have to correct for the fact that the time signal is dependent on the energy, or the amplitude of the signal, also often refered to as the problem of walk'. This problem originates form the use of a spesific type of discriminator. The discriminator gives a time signal, or a count, every time the amplitude are above a certain threshold. This threshold can either be fixed and independent on the signal, or it can be a fraction of the amplitude. The latter are called Constant Fracion Discriminator (CFD) which is the expensive version, and the first are called called Leading Edge Discriminators (LED) which are cheaper. In this experiment we have used LED which gives that the time signal is dependent on the energy. 

To correct for the 'problem of walk' we have done a curve fitting to the data of from all the NaI detectors, and coreccted for the energy dependence. In figure \ref{fig: time_corr} we see the uncorrected and corrected plot of the time channels versus the energy of all the NaI detectors.

\begin{figure}[htp]
\centering
\includegraphics[width=0.4\textwidth]{m_nai_e_t_corrected.pdf}
\includegraphics[width=0.4\textwidth]{m_nai_e_t_corrected_fit.pdf}
\caption{The time channels plotted versus the energy of the all NaI detectors. \textbf{Left:} uncalibrated plot with time signals shifted due to LED. \textbf{Right:} calibrated plot, corrected whith a curve fitting to the data in the left plot.}
\label{fig: time_corr}
\end{figure}

When the time signals are corrected we need to gate on the 'prompt gammas'. The 'prompt gammas' are the $\gamma$-rays that are in coincidence with the emitted proton. We can gate on the 'prompt gammas' by plotting the aligned time channels versus the number of counts. We then select the a  gate around the prompt peak so that we only select the experimental data corresponding to the desired reaction. 

But when we measure the $\gamma$-rays we might also measure delayed gammas, radiation from a previously excited nucleus that did not instantly decay (form the more stable state) or background. The background is $\gamma$-rays from other elements than the target that where excited. We can get rid of the backgrund by doing a measurment without target and then substract this to the experimental data. To get rid of the delayd gammas, or the random coincidences we set a gate on a peak corresponding to the random coincidences and subtract it to the gate on the prompt peak. When this is done we are finished with the $\gamma$ calibration.

%- Explain that we first aligned them and show the plot of the m_nai signals aligned. Explain that we have leading edge discriminators, and therefore we need to correct from "walk". Show the signals corrected and explain how you did this.
%- Selecting coincidences: show the plot and explain what is the prompt peak, what is the time resolution, and how we can calculate the time between beam pulses from the plot. Explain why we need to gate on the prompt peak.

\subsection{ The coincidence matrix}
When the data from the detectors have been corrected we are ready to plot the final goal; the coincidence matrix. In figure \ref{fig: coincidence} we see the coincidence matrix, the plot of the incident energy $E_x$ versus the $\gamma$-energy for different stages of the data analysis. The left plot is the coincidence matrix before the detector and time calibration, and the middle plot is after the calibration. To the right we find the final calibrated coincidence matrix with the background substracted. Due to the calibration we can see that the middle plot has clearer peaks than the left plot. All points under the diagonal $E_x=$E cannot correspond to the reaction because it implies that the incomming energy $E_x$ would be less than the resulting energy E. So it is expected that all the counts below this diagonal disappeares when we substract the background, which we can see is the case in the plot to the right. 

\begin{figure}[htp]
\centering
\includegraphics[width=0.3\textwidth]{m_alfna_week2.pdf}
\includegraphics[width=0.3\textwidth]{coincidence_wback}
\includegraphics[width=0.3\textwidth]{coincidence_noback}
\caption{The coincidence matrix (the incident energy $E_x$ versus the $\gamma$-energy) plotted for different stages of the data analysis. \textbf{Left:} before particle calibration. \textbf{Middle:} after particle calibration with background. \textbf{Right:} the final coincidence matrix, calibrated and with the background substracted.}
\label{fig: coincidence}
\end{figure}

%Show the coincidence matrix. Mention that there is background, how it can be measure and that it was subtracted. Show the diagonal Eg=Ex in the matrix. Why do we see some counts outside? Explain the peaks we see and comp.re them to the results from the nndc website or the TOI.

\section{The Oslo method}
noe noe tar lang tid

\subsection{Unfolding of the coincidence matrix}
%Explain the interactions of gamma with matter, and what does unfolding mean.
%Show the unfolded matrix.

\subsection{Multiplicity}


not gotten eny results really, need more analyzing to get any
\section{Discussion and Experiences}


\noindent\rule{\textwidth}{1pt}






\end{document}





