%%%%%%%%%%%%%%%%%%%%%%%%%%%%%%%%%%%%%%%%%
% Beamer Presentation
% LaTeX Template
% Version 1.0 (10/11/12)
%
% This template has been downloaded from:
% http://www.LaTeXTemplates.com
%
% License:
% CC BY-NC-SA 3.0 (http://creativecommons.org/licenses/by-nc-sa/3.0/)
%
%%%%%%%%%%%%%%%%%%%%%%%%%%%%%%%%%%%%%%%%%

%----------------------------------------------------------------------------------------
%	PACKAGES AND THEMES
%----------------------------------------------------------------------------------------

%\documentclass[notes]{beamer}       % print frame + notes
%\documentclass[notes=only]{beamer}   % only notes
\documentclass{beamer}              % only frames

\mode<presentation> {

% The Beamer class comes with a number of default slide themes
% which change the colors and layouts of slides. Below this is a list
% of all the themes, uncomment each in turn to see what they look like.

%\usetheme{default}
%\usetheme{AnnArbor}
%\usetheme{Antibes}
%\usetheme{Bergen}
%\usetheme{Berkeley}
%\usetheme{Berlin}
%\usetheme{Boadilla}
%\usetheme{CambridgeUS}
%\usetheme{Copenhagen}
%\usetheme{Darmstadt}
%\usetheme{Dresden}
%\usetheme{Frankfurt}
%\usetheme{Goettingen}
%\usetheme{Hannover}
%\usetheme{Ilmenau}
%\usetheme{JuanLesPins}
%\usetheme{Luebeck}
\usetheme{Madrid}
%\usetheme{Malmoe}
%\usetheme{Marburg}
%\usetheme{Montpellier}
%\usetheme{PaloAlto}
%\usetheme{Pittsburgh}
%\usetheme{Rochester}
%\usetheme{Singapore}
%\usetheme{Szeged}
%\usetheme{Warsaw}

% As well as themes, the Beamer class has a number of color themes
% for any slide theme. Uncomment each of these in turn to see how it
% changes the colors of your current slide theme.

%\usecolortheme{albatross}
%\usecolortheme{beaver}
%\usecolortheme{beetle}
%\usecolortheme{crane}
%\usecolortheme{dolphin}
%\usecolortheme{dove}
%\usecolortheme{fly}
%\usecolortheme{lily}
%\usecolortheme{orchid}
%\usecolortheme{rose}
%\usecolortheme{seagull}
%\usecolortheme{seahorse}
%\usecolortheme{whale}
%\usecolortheme{wolverine}

%\setbeamertemplate{footline} % To remove the footer line in all slides uncomment this line
%\setbeamertemplate{footline}[page number] % To replace the footer line in all slides with a simple slide count uncomment this line

\setbeamertemplate{navigation symbols}{} % To remove the navigation symbols from the bottom of all slides uncomment this line

%\setbeamertemplate{note page}[plain]
%\setbeameroption{show notes} %un-comment to see the notes

}
\usepackage{graphicx} % Allows including images
\usepackage{booktabs} % Allows the use of \toprule, \midrule and \bottomrule in tables
%\usepackage{epstopdf} %tried to include so .eps should work, but it did not, converted to pdf instead

\usepackage[T1]{fontenc} %for å bruke æøå
\usepackage[utf8]{inputenc}
\usepackage[norsk]{babel}
\newcommand{\btVFill}{\vskip0pt plus 1filll}

%----------------------------------------------------------------------------------------
%	TITLE PAGE
%----------------------------------------------------------------------------------------

\title{"An introduction to the experimental setup and analyzing methods used at the Oslo Cyclotron Laboratory by use of experimental data from the $\bf {}^{28}\mathrm{ \bf Si(p,p')}^{28}\mathrm{\bf Si}$ reaction."} 

\author{Ina Kullmann} % Your name
\institute[FI] % Your institution as it will appear on the bottom of every slide, may be shorthand to save space
{Department of Physics, University of Oslo \\ % Your institution for the title page
\medskip
\textit{ikkullma@student.matnat.uio.no} % Your email address
}
\date{\today} % Date, can be changed to a custom date

\begin{document}

\frame{\titlepage} % Print the title page as the first slide
\note{snakk om abstract}
%----------------------------------------------------------------------------------------
%	PRESENTATION SLIDES
%----------------------------------------------------------------------------------------

\frame{
\frametitle{Contents}\tableofcontents}
\note{snakk om siste avsnitt intro også}

%----------------------------------------------------------------------------------------
%----------------------------------------------------------------------------------------

\section{Motivation and purpose}
\begin{frame}
\frametitle{Motivation and purpose}
\begin{tabular}{cc}

\begin{minipage}{0.5\textwidth}
\medskip
\begin{figure}[htp]
\centering
\includegraphics[width=0.8\textwidth]{cactus_forside.jpg}
%\caption{The CACTUS detector at the Oslo Cyclotron Laboratory}% for the study of particle-$\gamma$ coincidences. (\texttt{//www.mn.uio.no/fysikk/english/research/about/infrastructure/OCL/}) }
\label{fig:front_page}
\end{figure}
\end{minipage}

\begin{minipage}{0.5\textwidth}
Applications:
\begin{itemize}
\item Astro physics
\item Reactor physics
\item Medical diagnosis and treatments
\end{itemize}
\note{snakk om introduction (første avsnittene)}
\end{minipage}

\end{tabular}

\end{frame}

%------------------------------------------------
%------------------------------------------------
\section{Experimental setup and method}

%\begin{frame}
%\frametitle{Experimental setup and method}
%%\textbf{The basic consepts of a cyclotron}
%%\linebreak
%%\linebreak
%\begin{figure}[htp]
%\centering
%\includegraphics[width=0.4\textwidth]{cyclotron_draw.jpg}
%\caption{A simple illustration of a cyclotron. }%\footnote{\texttt{http://www.mn.uio.no/fysikk/english/research/about/infrastructure/OCL/ocl-photos/}}}
%\label{fig:cyclotron_draw}
%\end{figure}
%\end{frame}

%------------------------------------------------
\begin{frame}
\frametitle{Experimental setup and method}
%The Oslo Cyclotron laboratory (OCL)
%\linebreak

\begin{figure}[htp]
\centering
\includegraphics[width=0.6\textwidth]{ocl-layout_mini.jpg}
\caption{An overview of the Oslo Cyclotron Laboratory.}% with the experimental hall to the right with the cyclotron at the bottom right. The beam line are indicated with a blue line and the target chamber is at the top left (CACTUS/SiRi). The possible beam types, energy and intensity ranges are indicated in the table to bottom left. }
\label{fig:OLC_exp_hall}
\end{figure}
\end{frame}

%------------------------------------------------
\begin{frame}
\frametitle{Experimental setup and method}

\begin{tabular}{cc}

\begin{minipage}{0.5\textwidth}
\begin{figure}[htp]
\centering
\includegraphics[width=\textwidth]{cactus_siri.png}
\caption{A incident particle hitting a target nucleus.}% The resulting emmited $\gamma$-ray is detected by the CACTUS detectors and the emmitted particle is detected by the SiRi detector.}% The angle between the incident trajectory and the trajectory of the emmitted particle is given as $\theta$. The two parts of the SiRi detector, 'dE' and 'E' is indicated in the figure.}
\label{fig: cactus_siri}
\end{figure}
\end{minipage}

\begin{minipage}{0.5\textwidth}
\begin{figure}[htp]
\centering
\includegraphics[width=0.9\textwidth]{siri2.png}
\caption{A picture of the Silicon Ring (SiRi).}
\label{fig: siri}
\end{figure}
\end{minipage}

\end{tabular}
\end{frame}

%------------------------------------------------
%\begin{frame}
%\frametitle{Experimental setup and method}
%%The SiRi detectors
%
%\end{frame}

%------------------------------------------------
\begin{frame}
\frametitle{Experimental setup and method}
%Choice of reaction: ${}^{28}\mathrm{Si(p,p')}^{28}\mathrm{Si}$
\begin{figure}[htp]
\centering
\includegraphics[width=0.5\textwidth]{reaction.png}
\caption{An illustration of the chosen ${}^{28}\mathrm{Si(p,p')}^{28}\mathrm{Si}$ reaction.}
\label{fig: reaction}
\end{figure}
\end{frame}

%------------------------------------------------
%------------------------------------------------
\section{Data analysis of the experimental data}
\begin{frame}
\frametitle{Data analysis of the experimental data}
\bigskip
\bigskip
\bigskip

\begin{tabular}{cc}

\begin{minipage}{0.5\textwidth}
\medskip
\begin{figure}[htp]
\centering
\includegraphics[width=0.9\textwidth]{data_nerd.jpg}
\end{figure}
\end{minipage}

\begin{minipage}{0.5\textwidth}
Some of the programs used:
\begin{itemize}
\item \texttt{Makefile} 
\item \texttt{User\_sort.cpp}
\item \texttt{Name.batch}
\begin{itemize}
\item \texttt{gainshifts525.dat}
\item \texttt{zrangep.dat}
\end{itemize}
\end{itemize}
\end{minipage}

\end{tabular}

\btVFill
\tiny{All source codes can be found at \texttt{https://github.com/oslocyclotronlab/oslo-method-software}.}

\note{
\begin{itemize}
\item \texttt{Makefile:} executable file that calls for example \texttt{User\_sort.cpp} and creates an executable called \texttt{sorting}.
\item \texttt{User\_sort.cpp:} the main sorting code in C++. It is possible to modify the code to include time gates, gates on excited nuclear states and so on. It defines what the executable \texttt{sorting} will do when it is run. 
\item \texttt{Sorting:} executable file created by the \texttt{Makefile}. It uses the batch file when run.
\item \texttt{Name.batch:} holds information on where to find the data files to be sorted. 'Name' is usually the experiment reaction. The file also includes several parameters and calls the following two programmes:
\begin{itemize}
\item \texttt{gainshifts525.dat:} contains information about the calibration of the particle and gamma detectors, together with the time signal calibration.
\item \texttt{zrangep.dat:} the range file for the ejected protons. It has information about how the ejected protons lose energy as they penetrate the E dE Si detectors.
\end{itemize}
\end{itemize}
}

\end{frame}

%------------------------------------------------
\begin{frame}
\frametitle{Data analysis of the experimental data}
%Particle calibration and bananas
\begin{figure}[htp]
\centering
\includegraphics[width=0.4\textwidth]{m_e_de_uncalibrated.png}
\includegraphics[width=0.45\textwidth]{m_e_de_calibrated.png} 
\caption{The 'particle bananas', or 'dE' versus 'E' plotted. \textbf{Left:} uncalibrated. \textbf{Right:} calibrated.}% Each banana corresponds to one emitted particle. We can also see the peacks corresponding to the excited states of the target nucleus.}
\label{fig: de_e}
\end{figure}
\end{frame}

%------------------------------------------------
\begin{frame}
\frametitle{Data analysis of the experimental data}
%Selecting the data for the ${}^{28}\mathrm{Si(p,p')}^{28}\mathrm{Si}$
\begin{figure}[htp]
\centering 
\includegraphics[width=0.5\textwidth]{h_thick_fromcalibrated_zoom.png}
\includegraphics[width=0.5\textwidth]{m_e_de_thick.png}
\caption{Selecting the data for the ${}^{28}$Si(p,p')$^{28}$Si reaction. \textbf{Left:} The calculated thickness of the $\Delta$E detector. \textbf{Right:} The 'banana' plot after particle calibration and gating on one particle banana.}
\label{fig: banana_gate}
\end{figure}
\end{frame}

%------------------------------------------------
\begin{frame}
\frametitle{Data analysis of the experimental data}
%Comparing with the NNDC database
%\linebreak
%\linebreak
\begin{tabular}{cc}

\begin{minipage}{0.5\textwidth}
\begin{figure}[htp]
\centering
\includegraphics[width=\textwidth]{h_ex_calibrated-zoom.png}
\caption{The coincidence matrix projected on the y axis giving number of counts versus energy, $E_x$. }%Each peak corresponds to a excited state of ${}^{28}$Si. The six first excited stated are marked with a red circle to be compared with table \ref{tab: e_levels}. }
\label{fig: proj_y_excitation}
\end{figure}
\end{minipage}

\begin{minipage}{0.5\textwidth}
\begin{table}
\centering
\caption{The energy levels of ${}^{28}$Si from the Chart of Nuclides found on the NNDC website.}% \texttt{//www.nndc.bnl.gov/chart/getdataset.jsp?nucleus=28SI\&unc=nds}. The uncertain digits are given in a parenthesis. }
\begin{tabular}{|c|}
\hline 
E\_x [keV] \\ 
\hline 
0 \\ 
\hline 
1779.030 (11) \\ 
\hline 
4617.86 (4) \\ 
\hline 
4979.92 (8) \\ 
\hline 
6276.2 (7) \\ 
\hline 
6690.74 (15) \\ 
\hline 
\end{tabular} 
\label{tab: e_levels}
\end{table}
\end{minipage}

\end{tabular}
\end{frame}

%------------------------------------------------
\begin{frame}
\frametitle{Data analysis of the experimental data: $\gamma$-calibration and treatment of the time signals}
\begin{figure}[htp]
\centering
\includegraphics[width=0.4\textwidth]{m_nai_e_t_corrected.png}
\includegraphics[width=0.5\textwidth]{m_nai_e_t_corrected_fit.png}
\caption{The time channels plotted versus the energy of the all NaI detectors. \textbf{Left:} uncalibrated plot with time signals shifted due to LED. \textbf{Right:} calibrated plot, corrected whith a curve fitting to the data in the left plot.}
\label{fig: time_corr}
\end{figure}
\end{frame}

%------------------------------------------------
\begin{frame}
\frametitle{Data analysis of the experimental data}
%The coincidence matrix
\begin{figure}[h!]
\centering
\includegraphics[width=0.75\textwidth]{m_alfna_week2.png}
%\includegraphics[width=0.5\textwidth]{coincidence_wback}
%\includegraphics[width=0.3\textwidth]{coincidence_noback}
\caption{The coincidence matrix, the excitation energy $E_x$ versus the $\gamma$-energy after particle, gamma and time calibration and with the background subtracted. }
\label{fig: coincidence}
\end{figure}
\end{frame}

%------------------------------------------------
%------------------------------------------------
\section{The Oslo method}
\begin{frame}
\frametitle{The Oslo method}
The Oslo method is based on three main steps:
\begin{enumerate}
\item Unfolding of the $\gamma$-ray spectra
\item Extraction of first-generation $\gamma$-rays
\item Extraction of the nuclear level density and the gamma strength function
\end{enumerate}
\begin{figure}
%\includegraphics[width=1\textwidth]{list.png}
\end{figure}
\end{frame}

%------------------------------------------------
\begin{frame}
\frametitle{The Oslo method:
Unfolding of the coincidence matrix}

\begin{tabular}{cc}
\begin{minipage}{0.5\textwidth}
The $\gamma$-matter interaction:
\begin{enumerate}
\item The photo electric effect
\item Compton scattering
\item Pair reaction (the gamma can produce a electron-positron pair and one or both can escape. 
\end{enumerate}
\end{minipage}

\begin{minipage}{0.5\textwidth}
\begin{figure}[htp]
\centering
\includegraphics[width=0.9\textwidth]{gamma_interaction.png}
%\caption{An illustration of the number of counts versus the gamma energy with the typical peaks obtained due to the photo electric effect, Compton scattering or pair reactions. }
\label{fig: gamma_interaction}
\end{figure}
\end{minipage}

\end{tabular}
\end{frame}

%------------------------------------------------
\begin{frame}
\frametitle{The Oslo method: Unfolding of the coincidence matrix}
The response matrix $\bf R$
\begin{figure}[htp]
\centering
\includegraphics[width=0.7\textwidth]{coincidence_noback}
\caption{The unfolded coincidence matrix, the excitation energy $E_x$ versus the $\gamma$-energy.}
\label{fig: un_coincidence}
\end{figure}
\end{frame}

%------------------------------------------------
\begin{frame}
\frametitle{The Oslo method: The extraction of the first-generation (primary) $\gamma$-rays emitted}

%\begin{equation}
%\langle M \rangle = c \cdot \frac{N_c}{N_s}
%\label{eq: M_iterative}
%\end{equation}

\begin{figure}[htp]
\centering
\includegraphics[width=0.45\textwidth]{gamma_decay.png}
\caption{An illustration of the possible paths a photon emitted from an excited state can take to the ground state. The secondary photons are marked with a red circle, the primary unmarked.}
\label{fig: gamma_decay}
\end{figure}
\end{frame}

%----------------------------------------------------------------------------------------
%----------------------------------------------------------------------------------------
\section{Discussion and experiences}
\begin{frame}
\frametitle{Discussion and experiences}

\begin{tabular}{cc}
\begin{minipage}{0.4\textwidth}
\begin{figure}[htp]
\centering
\includegraphics[width=\textwidth]{nebula2.png}
%\caption{An illustration of the possible paths a photon emitted from an excited state can take to the ground state. The secondary photons are marked with a red circle, the primary unmarked.}
\label{fig: nebula}
\end{figure}
\end{minipage}

\begin{minipage}{0.6\textwidth}
Learned a lot about:
\begin{itemize}
\item Experimental setup
\item The huge amount of 'data cleaning'
\item Why such experiments are useful 
\end{itemize}


\end{minipage}

\end{tabular}
\end{frame}

%----------------------------------------------------------------------------------------
\frame{\centering \textbf{Thank you!}}

\end{document} 
